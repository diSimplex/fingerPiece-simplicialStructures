% !TEX root = simpStruc.tex
% !LPiL preamble = ./ssPreamble.tex
% !LPiL postamble = ./ssPostamble.tex

\lpilTitle{fp-simpStruc}[
  Finger Pieces : Simplicial Structures
]{
  Finger Pieces : Simplicial Structures
}
\author{Stephen Gaito}

\maketitle

\begin{abstract}
  In this finger piece, we explore the Simplicial Structures used by the
  diSimplex projects.
\end{abstract}


\section*{Introduction}

Our ultimate objective is to provide a foundation for Quantum Relativity from
``first principles''. To do this we explicitly \emph{do not} assume the
existence of any background or ambient ``space'' or manifold.

Initially we only assume the existence of a collection (set) of ``events''.
Later when we want to study a ``relativistic'' structure we will assume that
the ``events'' are ``partially ordered''.

To be able to provide a foundation for Quantum Relativity, we need to,
\emph{eventually}, provide the ``standard tools'' of both General Relativity as
well as Quantum Field Theory. Both General Relativity as well as Quantum Field
Theory must end up being ``sectors'' of ``Quantum Relativity''. This means that
based \emph{only} on the collection of events, we need to be able to develop the
``normal'' machinery of Mathematical Field Theory. That is, for example, we need
to be able to construct Exterior Forms as well as Lie Groups associated with our
``manifolds''.

\section*{Functors} 

Categorical tools will be the heart of everything we do to describe Simplicial
Structures. Each Simplicial Structure is its own Category. The collection of all
Simplicial Structures is also a Category (of Categories). However, we will use a
fairly standard ``presheaf'' approach to defining Simplicial Structures. 

We begin by looking at the category of order preserving maps between the natural
numbers. 

